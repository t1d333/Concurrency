\documentclass[a4paper, 14pt]{extarticle}

% Поля
%----------------------
\usepackage{geometry}
\geometry{a4paper,left=2cm,right=1cm,
    top=2cm,bottom=2cm,bindingoffset=0cm}
%----------------------

% Russian-specific packages
%----------------------
\usepackage[T2A]{fontenc}
\usepackage[utf8]{inputenc}
\usepackage[english, main=russian]{babel}
%----------------------

\usepackage{textcomp}

% Красная строка
%----------------------
\usepackage{indentfirst}
%----------------------

% Graphics
%----------------------
\usepackage{graphicx}
\graphicspath{ {./images} }
\usepackage{wrapfig}
%----------------------

% Import minted
%----------------------
\usepackage{minted}
%----------------------

% Import pgfplots
%----------------------
\usepackage{pgfplots}
\pgfplotsset{width=10cm,compat=newest}
%----------------------

\linespread{1.3}
\sloppy
\clubpenalty=10000
\widowpenalty=10000

\begin{document}

%--------------------------------------
%			ТИТУЛЬНЫЙ ЛИСТ
%--------------------------------------
\begin{titlepage}
\thispagestyle{empty}
\newpage


%Шапка титульного листа
%--------------------------------------
\vspace*{-60pt}
\hspace{-65pt}
\begin{minipage}{0.3\textwidth}
\hspace*{-20pt}\centering
\includegraphics[width=\textwidth]{emblem}
\end{minipage}
\begin{minipage}{0.67\textwidth}\small \textbf{
\vspace*{-0.7ex}
\hspace*{-6pt}\centerline{Министерство науки и высшего образования Российской Федерации}
\vspace*{-0.7ex}
\centerline{Федеральное государственное бюджетное образовательное учреждение }
\vspace*{-0.7ex}
\centerline{высшего образования}
\vspace*{-0.7ex}
\centerline{<<Московский государственный технический университет}
\vspace*{-0.7ex}
\centerline{имени Н.Э. Баумана}
\vspace*{-0.7ex}
\centerline{(национальный исследовательский университет)>>}
\vspace*{-0.7ex}
\centerline{(МГТУ им. Н.Э. Баумана)}}
\end{minipage}
%--------------------------------------

%Полосы
%--------------------------------------
\vspace{-25pt}
\hspace{-35pt}\rule{\textwidth}{2.3pt}

\vspace*{-20.3pt}
\hspace{-35pt}\rule{\textwidth}{0.4pt}
%--------------------------------------

\vspace{1.5ex}
\hspace{-35pt} \noindent \small ФАКУЛЬТЕТ\hspace{80pt} <<Информатика и системы управления>>

\vspace*{-16pt}
\hspace{47pt}\rule{0.83\textwidth}{0.4pt}

\vspace{0.5ex}
\hspace{-35pt} \noindent \small КАФЕДРА\hspace{50pt} <<Теоретическая информатика и компьютерные технологии>>

\vspace*{-16pt}
\hspace{30pt}\rule{0.866\textwidth}{0.4pt}
  
\vspace{11em}

\begin{center}
\Large {\bf Лабораторная работа №2 } \\ 
\large {\bf по курсу <<Разработка параллельных и распределенных программ>>} \\

% Labname
\large << Параллельная реализация решения системы линейных алгебраических
уравнений с помощью MPI>> \\
\end{center}\normalsize

\vspace{8em}


\begin{flushright}
  {Студент группы ИУ9-51Б Киселев К.А. \hspace*{15pt}\\ 
  \vspace{2ex}
  Преподаватель Царев А.С.\hspace*{15pt}}
\end{flushright}

\bigskip

\vfill
 

\begin{center}
\textsl{Москва 2023}
\end{center}
\end{titlepage}
%--------------------------------------
%		КОНЕЦ ТИТУЛЬНОГО ЛИСТА
%--------------------------------------

\newpage

\tableofcontents

\newpage

\section{Постановка задачи}

\begin{enumerate}
	\item{Написать программу, которая реализует итерационный алгоритм
		решения системы линейных алгебраических уравнений вида Ax=b в
		соответствии с выбранным вариантом. Здесь A – матрица размером
		N×N, x и b – векторы длины N. Тип элементов – double.
		}

	\item{
		Программу распараллелить с помощью MPI с разрезанием матрицы A
		по строкам на близкие по размеру, возможно не одинаковые, части.
		Соседние строки матрицы должны располагаться в одном или в
		соседних MPI-процессах. Реализовать два варианта программы:
		1: векторы x и b дублируются в каждом MPI-процессе,
		2: векторы x и b разрезаются между MPI-процессами аналогично
		матрице A. (только для сдающих после срока)
		Уделить внимание тому, чтобы при запуске программы на различном
		числе MPI-процессов решалась одна и та же задача (исходные данные
		заполнялись одинаковым образом).
		}
		
	\item{Замерить время работы двух вариантов программы при использовании
		различного числа процессорных ядер: 1,2, 4, 8, 16. Построить графики
		зависимости времени работы программы, ускорения и эффективности
		распараллеливания от числа используемых ядер. Исходные данные,
		параметры N и $\varepsilon$ подобрать таким образом, чтобы решение задачи на
		одном ядре занимало не менее 30 секунд. Также параметр N разрешено
		подобрать таким образом, чтобы он нацело делился на на 1,2,4,8 и 16.
		}

	\item{На основании полученных результатов сделать вывод о
		целесообразности использования одного или второго варианта
		программы.
		}
\end{enumerate}


\newpage

\section{Практическая реализация}


Код программы для решение СЛАУ методом простых итераций с делением матрицы $A$ между процессами

\begin{minted}
[
frame=lines,
framesep=2mm,
baselinestretch=1.2,
fontsize=\footnotesize,
linenos,
breakanywhere,
tabsize=2
]
{golang}

func (s *Solver) FindSolution() *mat.VecDense {
	res := mat.NewVecDense(s.size, nil)
	metric := math.MaxFloat64

	prev := metric

	buff := make([]float64, s.size)
	for metric > s.eps {
		s.comm.AllGatherF64(buff, s.iterateSolution(res).RawVector().Data)
		tmp := mat.NewVecDense(s.size, buff)

		res.SubVec(res, tmp)

		prev = metric
		metric = s.calcMetric(res)

		if prev < metric {
			s.tau *= -1
		}

	}

	return res
}

func (s *Solver) iterateSolution(chunk *mat.VecDense) *mat.VecDense {
	tmp := mat.NewVecDense(s.end-s.start, nil)

	tmp.MulVec(s.coeffs, chunk)
	tmp.SubVec(tmp, s.freeCoeffs.SliceVec(s.start, s.end))
	tmp.ScaleVec(s.tau, tmp)

	return tmp
}

func (s *Solver) calcMetric(xVec *mat.VecDense) float64 {
	tmp := mat.NewVecDense(s.end-s.start, nil)
	tmp.MulVec(s.coeffs, xVec)
	tmp.SubVec(tmp, s.freeCoeffs.SliceVec(s.start, s.end))

	frac := make([]float64, 2)
	num := float64(0)
	den := float64(0)

	for _, v := range tmp.RawVector().Data {
		num += v * v
	}

	for _, v := range s.freeCoeffs.RawVector().Data {
		den += v * v
	}

	s.comm.AllReduceF64(mpi.OpSum, frac, []float64{num, den})

	return math.Sqrt(frac[0]) / math.Sqrt(frac[1])
}
\end{minted}


\newpage

Код программы для решение СЛАУ методом простых итераций с делением матрицы $A$ и векторов $x, b$ между процессами

\begin{minted}
[
frame=lines,
framesep=2mm,
baselinestretch=1.2,
fontsize=\footnotesize,
linenos,
breakanywhere,
tabsize=2
]
{golang}

func (s *SolverWithVecSeparation) FindSolution() *mat.VecDense {
	res := mat.NewVecDense(s.end-s.start, nil)
	metric := math.MaxFloat64

	prev := metric

	buff := make([]float64, s.size)
	for metric > s.eps {
		chunk := s.iterateSolution(res).RawVector().Data

		s.comm.AllGatherF64(buff, chunk)
		tmp := mat.NewVecDense(s.size, buff)

		res.SubVec(res, tmp.SliceVec(s.start, s.end))

		prev = metric
		metric = s.calcMetric(res)

		if prev < metric {
			s.tau *= -1
		}

	}

	return res
}

func (s *SolverWithVecSeparation) iterateSolution(chunk *mat.VecDense) *mat.VecDense {
	buff := make([]float64, s.size)
	s.comm.AllGatherF64(buff, chunk.RawVector().Data)

	xvec := mat.NewVecDense(s.size, buff)

	tmp := mat.NewVecDense(s.end-s.start, nil)

	tmp.MulVec(s.coeffs, xvec)
	tmp.SubVec(tmp, s.freeCoeffs)
	tmp.ScaleVec(s.tau, tmp)

	return tmp
}

func (s *SolverWithVecSeparation) calcMetric(chunk *mat.VecDense) float64 {
	buff := make([]float64, s.size)

	s.comm.AllGatherF64(buff, chunk.RawVector().Data)

	xvec := mat.NewVecDense(s.size, buff)

	tmp := mat.NewVecDense(s.end-s.start, nil)

	tmp.MulVec(s.coeffs, xvec)
	tmp.SubVec(tmp, s.freeCoeffs)

	frac := make([]float64, 2)
	num := float64(0)
	den := float64(0)

	for _, v := range tmp.RawVector().Data {
		num += v * v
	}

	for _, v := range s.freeCoeffs.RawVector().Data {
		den += v * v
	}

	s.comm.AllReduceF64(mpi.OpSum, frac, []float64{num, den})

	return math.Sqrt(frac[0]) / math.Sqrt(frac[1])
}
\end{minted}

\newpage

\section{Результаты сравнения производительности}

\begin{figure}[H]
	\begin{tikzpicture}

	\begin{axis}[
		height=9cm,
		width=0.9\textwidth,
		grid=major,
		xmin = 0,
		xmax = 5,
		xtick distance=1,
		xlabel={количество процессов},
		ylabel={время выполнения, сек},
	]
		
	\addplot[color=blue,mark=*] coordinates {
		(1,41)
		(2,29)
		(3,25)
		(4,23)
	};
	\addlegendentry{C делением матрицы коэф-тов}	
		
	\addplot[color=red,mark=*] coordinates {
		(1,46)
		(2,32)
		(3,25)
		(4,24)
	};
	\addlegendentry{С делением матрицы коэф-тов и векторов}	
	\end{axis}%
\end{tikzpicture}%
	\caption{зависимость времени выполнения от количества процессов(n=36000)}
\end{figure}


\end{document}
