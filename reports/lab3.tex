\documentclass[a4paper, 14pt]{extarticle}

% Поля
%----------------------
\usepackage{geometry}
\geometry{a4paper,left=2cm,right=1cm,
    top=2cm,bottom=2cm,bindingoffset=0cm}
%----------------------

% Russian-specific packages
%----------------------
\usepackage[T2A]{fontenc}
\usepackage[utf8]{inputenc}
\usepackage[english, main=russian]{babel}
%----------------------

\usepackage{textcomp}

% Красная строка
%----------------------
\usepackage{indentfirst}
%----------------------

% Graphics
%----------------------
\usepackage{graphicx}
\graphicspath{ {./images} }
\usepackage{wrapfig}
%----------------------

% Import minted
%----------------------
\usepackage{minted}
%----------------------

% Import pgfplots
%----------------------
\usepackage{pgfplots}
\pgfplotsset{width=10cm,compat=newest}
%----------------------

\linespread{1.3}
\sloppy
\clubpenalty=10000
\widowpenalty=10000

\begin{document}

%--------------------------------------
%			ТИТУЛЬНЫЙ ЛИСТ
%--------------------------------------
\begin{titlepage}
\thispagestyle{empty}
\newpage


%Шапка титульного листа
%--------------------------------------
\vspace*{-60pt}
\hspace{-65pt}
\begin{minipage}{0.3\textwidth}
\hspace*{-20pt}\centering
\includegraphics[width=\textwidth]{emblem}
\end{minipage}
\begin{minipage}{0.67\textwidth}\small \textbf{
\vspace*{-0.7ex}
\hspace*{-6pt}\centerline{Министерство науки и высшего образования Российской Федерации}
\vspace*{-0.7ex}
\centerline{Федеральное государственное бюджетное образовательное учреждение }
\vspace*{-0.7ex}
\centerline{высшего образования}
\vspace*{-0.7ex}
\centerline{<<Московский государственный технический университет}
\vspace*{-0.7ex}
\centerline{имени Н.Э. Баумана}
\vspace*{-0.7ex}
\centerline{(национальный исследовательский университет)>>}
\vspace*{-0.7ex}
\centerline{(МГТУ им. Н.Э. Баумана)}}
\end{minipage}
%--------------------------------------

%Полосы
%--------------------------------------
\vspace{-25pt}
\hspace{-35pt}\rule{\textwidth}{2.3pt}

\vspace*{-20.3pt}
\hspace{-35pt}\rule{\textwidth}{0.4pt}
%--------------------------------------

\vspace{1.5ex}
\hspace{-35pt} \noindent \small ФАКУЛЬТЕТ\hspace{80pt} <<Информатика и системы управления>>

\vspace*{-16pt}
\hspace{47pt}\rule{0.83\textwidth}{0.4pt}

\vspace{0.5ex}
\hspace{-35pt} \noindent \small КАФЕДРА\hspace{50pt} <<Теоретическая информатика и компьютерные технологии>>

\vspace*{-16pt}
\hspace{30pt}\rule{0.866\textwidth}{0.4pt}
  
\vspace{11em}

\begin{center}
\Large {\bf Лабораторная работа №3 } \\ 
\large {\bf по курсу <<Разработка параллельных и распределенных программ>>} \\

% Labname
\large << Параллельная реализация решения системы линейных алгебраических
уравнений с помощью OpenMP>> \\
\end{center}\normalsize

\vspace{8em}


\begin{flushright}
  {Студент группы ИУ9-51Б Киселев К.А. \hspace*{15pt}\\ 
  \vspace{2ex}
  Преподаватель Царев А.С.\hspace*{15pt}}
\end{flushright}

\bigskip

\vfill
 

\begin{center}
\textsl{Москва 2023}
\end{center}
\end{titlepage}
%--------------------------------------
%		КОНЕЦ ТИТУЛЬНОГО ЛИСТА
%--------------------------------------

\newpage

\tableofcontents

\newpage

\section{Постановка задачи}

\begin{enumerate}
	\item{Написать программу, которая реализует итерационный алгоритм
		решения системы линейных алгебраических уравнений вида Ax=b в
		соответствии с выбранным вариантом. Здесь A – матрица размером
		N×N, x и b – векторы длины N. Тип элементов – double.
		}

	\item{
		Программу распараллелить с помощью MP с разрезанием матрицы A
		по строкам на близкие по размеру, возможно не одинаковые, части.
		Соседние строки матрицы должны располагаться в одном или в
		соседних MPI-процессах. Реализовать два варианта программы:
		1: векторы x и b дублируются в каждом потоке,
		2: векторы x и b разрезаются между потоками аналогично
		матрице A. (только для сдающих после срока)
		Уделить внимание тому, чтобы при запуске программы на различном
		числе потоков решалась одна и та же задача (исходные данные
		заполнялись одинаковым образом).
		}
		
	\item{Замерить время работы двух вариантов программы при использовании
		различного числа процессорных ядер: 1,2, 4, 8, 16. Построить графики
		зависимости времени работы программы, ускорения и эффективности
		распараллеливания от числа используемых ядер. Исходные данные,
		параметры N и $\varepsilon$ подобрать таким образом, чтобы решение задачи на
		одном ядре занимало не менее 30 секунд. Также параметр N разрешено
		подобрать таким образом, чтобы он нацело делился на на 1,2,4,8 и 16.
		}

	\item{На основании полученных результатов сделать вывод о
		целесообразности использования одного или второго варианта
		программы.
		}
\end{enumerate}


\newpage

\section{Практическая реализация}


Код программы для решение СЛАУ методом простых итераций с делением матрицы $A$ между потоками  

\begin{minted}
[
frame=lines,
framesep=2mm,
baselinestretch=1.2,
fontsize=\footnotesize,
linenos,
breakanywhere,
tabsize=2
]
{c++}

#include "solver.hpp"
#include "boost/numeric/ublas/matrix.hpp"
#include <boost/numeric/ublas/io.hpp>
#include <boost/numeric/ublas/matrix_proxy.hpp>
#include <boost/numeric/ublas/vector.hpp>
#include <boost/numeric/ublas/vector_proxy.hpp>
#include <cmath>
#include <omp.h>

using namespace boost::numeric;

const double EPSILON = 1e-4;

Solver::Solver(size_t size, ublas::matrix<double> &coeffs,
               ublas::vector<double> &free_coeffs)
    : size(size) {
  this->coeffs = ublas::matrix<double>(coeffs);
  this->free_coeffs = ublas::vector<double>(free_coeffs);
}

ublas::vector<double> Solver::FindSolution() {

  ublas::vector<double> result(size);

  double tau = double(1) / this->size;
  double metric = std::numeric_limits<double>::max();
  double prev = metric;

  while (metric >= EPSILON) {
    prev = metric;

#pragma omp parallel shared(result)
    {
      size_t thread_num = omp_get_thread_num();
      size_t chunk_size = this->size / omp_get_max_threads();
      size_t start_index = thread_num * chunk_size;
      size_t end_index = (thread_num == omp_get_max_threads() - 1)
                             ? this->size
                             : start_index + chunk_size;

      ublas::vector<double> chunk = this->iterate_solution(result, tau);

#pragma omp barrier
      for (size_t j = start_index; j < end_index; ++j) {
        result(j) -= chunk(j - start_index);
      }

#pragma omp barrier
    }

    metric = this->calc_metric(result);

    if (prev < metric) {
      tau *= -1;
    }
  }

  return result;
}

ublas::vector<double> Solver::iterate_solution(ublas::vector<double> &sol,
                                               double tau) {
  ublas::vector<double> tmp_free = ublas::subslice(
      this->free_coeffs,
      omp_get_thread_num() * (this->size / omp_get_max_threads()), 1,
      this->size / omp_get_max_threads());

  ublas::matrix<double> tmp_coeffs = ublas::subslice(
      this->coeffs, omp_get_thread_num() * (this->size / omp_get_max_threads()),
      1, this->size / omp_get_max_threads(), 0, 1, this->size);
  ublas::vector<double> tmp = ublas::prod(tmp_coeffs, sol);
  tmp -= tmp_free;
  tmp *= tau;
  return tmp;
}

double Solver::calc_metric(ublas::vector<double> &sol) {

  double numerator = 0;
#pragma omp parallel shared(numerator)
  {
    ublas::vector<double> tmp_free = ublas::subslice(
        this->free_coeffs,
        omp_get_thread_num() * (this->size / omp_get_max_threads()), 1,
        this->size / omp_get_max_threads());

    ublas::matrix<double> tmp_coeffs = ublas::subslice(
        this->coeffs,
        omp_get_thread_num() * (this->size / omp_get_max_threads()), 1,
        this->size / omp_get_max_threads(), 0, 1, this->size);
    ublas::vector<double> tmp = ublas::prod(tmp_coeffs, sol);
    tmp -= tmp_free;

    double sum = 0;
    std::for_each(tmp.begin(), tmp.end(), [&sum](double d) { sum += d * d; });
#pragma omp atomic
    numerator += sum;
  }

  return std::sqrt(numerator) / ublas::norm_2(this->free_coeffs);
}
\end{minted}


\newpage

\section{Результаты сравнения производительности}

\begin{figure}[H]
	\begin{tikzpicture}

	\begin{axis}[
		height=9cm,
		width=0.9\textwidth,
		grid=major,
		xmin = 0,
		xmax = 66,
		xtick distance=8,
		xlabel={количество потоков},
		ylabel={время выполнения, сек},
	]
		
	\addplot[color=blue,mark=*] coordinates {
		(1, 107)
		(2, 54)
		(4, 32)
		(8, 30)
		(16, 28)
		(32, 30)
		(64, 30)
	};
		
	\end{axis}%
\end{tikzpicture}%
	\caption{зависимость времени выполнения от количества потоков}
\end{figure}


\end{document}
